\section{Harun Ar - Rasyid}
\begin{enumerate}
    \item Apa itu fungsi file csv, jelaskan sejarah dan contoh
    File CSV (Nilai Terbatas Koma) adalah jenis file khusus yang dapat Anda buat atau edit di Excel. File CSV menyimpan informasi yang dipisahkan oleh koma, tidak menyimpan informasi dalam kolom. Ketika teks dan angka disimpan dalam file CSV, mudah untuk memindahkannya dari satu program ke program lainnya.
    Dari rilis pertama, Excel menggunakan format file biner yang disebut Binary Interchange File Format (BIFF) sebagai format file utamanya. Ini berubah ketika Microsoft merilis Office System 2007 yang memperkenalkan Office Open XML sebagai format file utamanya. Office Open XML adalah file kontainer berbasis XML yang mirip dengan XML Spreadsheets (XMLSS), yang diperkenalkan di Excel 2002. File versi XML tidak bisa menyimpan makro VBA.
    Meskipun mendukung format XML baru, Excel 2007 masih mendukung format lama yang masih berbasis BIFF tradisional. Selain itu Microsoft Excel juga mendukung format Comma Separated Values (CSV), DBase File (DBF), SYMbolic LinK (SYLK), Format Interchange Data (DIF) dan banyak format lainnya, termasuk format lembar kerja 1-2 Lotus - 3 (WKS, WK1, WK2, dll.) Dan Quattro Pro.
    \item Aplikasi-aplikasi apa saja yang bisa menciptakan file csv
    \begin{itemize}
        \item Texteditor
        Seperti notepad++,visual studio code,atom,sublime dan lain sebagainya
        \item Program Spreadsheet
        Seperti excell,google spreadshare,LibreOfficecalc
    \end{itemize}
    \item Jelaskan bagaimana cara menulis dan membaca file csv di excel atau spreadsheet
    Untuk menulisnya untuk yang paling atas itu kita buat headernya,untuk mepermudah membedakan datanya,dan untuk baris kedua dan seterusnya itu untuk data itu sendiri.
    dan setelah di buat kalian save as kemudian pilih format CSV.
    dan untuk membukan cukup di double clik file tersebut
    \item Jelaskan sejarah library csv
    library csv dibuat untuk permudah mengolah data. Dan mempermudah untuk melakukan export dan import file csv itu sendiri
    \item Jelaskan sejarah library pandas
    library pandas dibuat agar bahasa pemograman python bisa bersaing R dan matlab, yang digunakan untuk mengolah banyak data , keperluan big data, data mining data science dan sebagainya.
    \item Jelaskan fungsi-fungsi yang terdapat di library csv
    Terdapat 2 fungsi yang bisa digunakan oleh library csv
    Pertama,fungsi membaca file csv.
    fungsi ini bisa menggunakan list dan dictionary
    Dengan list :
    \lstinputlisting[firstline=11, lastline=21]{src/1174027/1174027_csv.py}
    Dengan dictionary :
    \lstinputlisting[firstline=24, lastline=33]{src/1174027/1174027_csv.py}
    Kedua,fungsi menulis file csv.
    \lstinputlisting[firstline=36, lastline=40]{src/1174027/1174027_csv.py}
    \item Jelaskan fungsi-fungsi yang terdapat di library pandas
    Hampir sama dengan library csv,tp library pandas penulisannya lebih sederhana dan terlihat lebih rapih dari pada library csv.
    \lstinputlisting[firstline=43, lastline=44]{src/1174027/1174027_csv.py}
\end{enumerate}
%%%%%%%%%%%%%%%%%%%%%%%%%%%%%%%%%%%%%%%%%%%%%%%%%%%%%%%%

\section{Dwi Yulianingsih}
\subsection{Pemahaman Materi}
\begin{enumerate}
\item Apa itu fungsi file csv, jelaskan sejarah dan contoh
\paragraph{} CSV (Comma Separated Value) adalah format basis data sederhana yang dimana setiap record yang ada dipisahkan dengan tanda koma (,) atau titik koma (;). Format data file csv dapat diolah dengan berbagai text editor dengan mudah. Anda tidak perlu (dan Anda tidak akan) membuat pengurai CSV Anda sendiri dari awal. Ada beberapa perpustakaan yang dapat diterima yang dapat Anda gunakan. Pustaka csv Python akan berfungsi untuk sebagian besar kasus. Jika pekerjaan Anda memerlukan banyak data atau analisis numerik, panda library juga memiliki kemampuan penguraian CSV, yang seharusnya menangani sisanya. Dalam bahasa pemrograman Python telah disediakan modul csv yang khusus untuk mengolah data berformat csv.  Untuk memanipulasi data csv dengan python tentunya yang pertama dilakukan adalah mengimport modul csv dengan perintah import csv. File CSV biasanya dibuat oleh program yang menangani sejumlah besar data. Mereka adalah cara yang nyaman untuk mengekspor data dari spreadsheet dan basis data serta mengimpor atau menggunakannya dalam program lain. Misalnya, Anda dapat mengekspor hasil program penambangan data ke file CSV dan kemudian mengimpornya ke dalam spreadsheet untuk menganalisis data, menghasilkan grafik untuk presentasi, atau menyiapkan laporan untuk publikasi. Contoh nya adalah sebagai berikut :

 \lstinputlisting[firstline=8, lastline=20]{src/1174009/dudul.py}

\item Aplikasi-aplikasi apa saja yang bisa menciptakan file csv?
\paragraph{} Ada beberapa aplikasi yang dapat menciptakan file dengan format csv diantaranya google sheet, number di MacOS dan microsoft excel.
\subsection{Membuat dan membaca csv di excel atau spreadsheet}

\item Jelaskan bagaimana cara menulis dan membaca file csv di excel atau spreadsheet
\paragraph{} Cara membuat file csv di excel cukup mudah yaitu :
\begin{itemize}
	\item Buat foldernya
	\item Pilih save as
	\item pilih file dengan format csv
\end{itemize}
Cara membaca file di csv :
\begin{itemize}
	\item Klik data - get external data - form text
	\item Akan muncul Text Import Wizard, arahkan pada file csv yang ingin anda buka lalu Open.
	\item Setelah File terbuka, akan muncul Text Import Wizard.
	\item Pilih Delimited, Kemudian Next (Di sini, bisa juga menentukan baris awal yang akan di import)
	\item Centrang pada Tab dan Comma (Atau sesuai pengaturan File Anda) lalu Next.
	\item Atur Format data pada tiap kolom yang tampil dan klik Finish
\end{itemize}

\item Jelaskan sejarah library csv
\paragraph{} CSV muncul untuk memudahkan data science dan analis karena dinilai terdapat banyak kemudahan yang didapat. CSV dapat dimaksimalkan jika dipaduka dengan python karena python adalah bahasa pemrograman yang support ke banyak library termasuk csv. Maka karena itulah perpaduan python dan csv seringkali digunakan oleh perusahaan-perushaan besar dalam mengolah datanya.

\item Jelaskan sejarah library pandas
\paragraph{} Pandas merupakan tool yang dapat digunakan sebagai alat analisis data dan struktur untuk bahasa pemrograman Python. Pandas dapat mengolah data dengan mudah, salah satu fitur yang ada dalam pandas adalah Dataframe. Fitur dataframe dapat membaca sebuah file dan menjadikannya tabble, juga dapat mengolah suatu data dengan menggunakan operasi seperti join, group by dan teknik lainnya yang terdapat pada SQL. Dalam hal ini pandas tidak jauh beda dengan csv yaitu memiliki keunggulan dalam pengolahan data-data besar dan dapat disupport dengan baik dengan python walaupun mengimport data dalam jumlah banyak.

\item Jelaskan fungsi-fungsi yang terdapat di library csv
\paragraph{} Library csv mempunyai keunggulan dibandingkan format data lainnya adalah soal kompatibilitas. File csv dapat digunakan, diolah, diekspor/impor, dan dimodifikasi menggunakan berbagai macam perangkat lunak dan bahasa pemrograman. Pada library csv mempunyai fungsi import dan eksport data yang baik dan bisa digunakan dalam jumlah besar.

\item Jelaskan fungsi-fungsi yang terdapat di library pandas
\paragraph{} pandas menyediakan beragam fungsi operasi untuk mengolah data. Contoh jika menggunakan series bisa mencari nilai max, min, dan mean secara langsung, bahkan juga bisa melakukan operasi perpangkatan pada nilai Series secara langsung.
Pandas dapat mengolah suatu data dan mengolahnya seperti join, distinct, group by, agregasi, dan teknik seperti pada SQL. Hanya saja dilakukan pada tabel yang dimuat dari file ke RAM.
\end{enumerate}

\subsection{bukti bebas plagiarisme}
\begin{figure}[H]
\centering
\includegraphics[width=10cm]{figures/yuli.png}
\caption{SS Bebas Plagiarisme}
\label{dwiyul}
\end{figure}
%%%%%%%%%%%%%%%%%%%%%%%%%%%%%%%%%%%%%%%%%%%%%%%%%%%%%%%%

\section{Muhammad Dzihan Al-Banna}
\subsection{Sejarah Csv}
\paragraph{}Comma Separated Value atau CSV adalah format data yang memudahkan penggunanya melakukan input data ke database secara sederhana. CSV dapat digunakan dalam standar file ASCII. Dalam format csv record dipisahkan dengan tanda koma atau titik koma. Ketika user menerima file dengan format CSV, yang biasanya bertuliskan .CSV, maka file tersebut akan terbuka dalam format Microsoft Excel. CSV muncul demi memenuhi kebutuhan perusahaan-perusahaan besar dalam mengolah data yang banyak.
\lstinputlisting[firstline=7, lastline=20]{src/1174095/cobacsv.py}
\subsubsection{Fungsi CSV}
\paragraph{}Fungsi csv yaitu memudahkan user dalam melakukan input data karena di csv input data atau import data dalam skala besar dapat dilakukan dengan cara yang sederhana.
\subsection{Aplikasi yang dapat menghasilkan csv}
\paragraph{}Ada beberapa aplikasi yang dapat menghasilkan file dengan format csv diantaranya google sheet, number di MacOS dan microsoft excel.
\subsection{Membuat dan membaca csv di excel atau spreadsheet}
\subsubsection{Membuat dan membaca csv di excel}
cara membuat file csv di excel cukup mudah yaitu :
\begin{itemize}
	\item Buat foldernya
	\item Pilih save as
	\item pilih file dengan format csv
\end{itemize}
cara membaca file di csv :
\begin{itemize}
	\item Klik data get external data form text
	\item Akan muncul Text Import Wizard, arahkan pada file csv yang ingin anda buka Open.
	\item Setelah File terbuka, akan muncul Text Import Wizard.
	\item Pilih Delimited, Kemudian Next (Di sini, bisa juga menentukan baris awal yang akan di import)
	\item Centrang pada Tab dan Comma (Atau sesuai pengaturan File Anda) Next.
	\item Atur Format data pada tiap kolom yang tampil dan klik Finish
\end{itemize}
\subsection{Sejarah Library CSV}
\paragraph{}CSV muncul untuk memudahkan data science dan analis karena dinilai terdapat banyak kemudahan yang didapat. CSV dapat dimaksimalkan jika dipaduka dengan python karena python adalah bahasa pemrograman yang support ke banyak library termasuk csv. Maka karena itulah perpaduan python dan csv seringkali digunakan oleh perusahaan-perushaan besar dalam mengolah datanya.
\subsection{Sejarah Library Pandas}
\paragraph{}Pandas merupakan tool yang dapat digunakan sebagai alat analisis data dan struktur untuk bahasa pemrograman Python. Pandas dapat mengolah data dengan mudah, salah satu fitur yang ada dalam pandas adalah Dataframe. Fitur dataframe dapat membaca sebuah file dan menjadikannya tabble, juga dapat mengolah suatu data dengan menggunakan operasi seperti join, group by dan teknik lainnya yang terdapat pada SQL. Dalam hal ini pandas tidak jauh beda dengan csv yaitu memiliki keunggulan dalam pengolahan data-data besar dan dapat disupport dengan baik dengan python walaupun mengimport data dalam jumlah banyak.
\subsection{Fungsi-fungsi Library CSV}
\paragraph{}Dalam library csv terdapat dua fungsi yaiut fungsi membaca file dan menulis file csv.
Library csv mempunyai keunggulan dibandingkan format data lainnya adalah soal kompatibilitas. File csv dapat digunakan, diolah, diekspor/impor, dan dimodifikasi menggunakan berbagai macam perangkat lunak dan bahasa pemrograman. Pada library csv mempunyai fungsi import dan eksport data yang baik dan bisa digunakan dalam jumlah besar.
\subsection{Fungsi-fungsi library Pandas}
\paragraph{}Pandas pun memiliki fungsi yang sama yaitu menulis dan membaca file. pandas menyediakan beragam fungsi operasi untuk mengolah data. Contoh jika menggunakan series bisa mencari nilai max, min, dan mean secara langsung, bahkan juga bisa melakukan operasi perpangkatan pada nilai Series secara langsung.
Pandas dapat mengolah suatu data dan mengolahnya seperti join, distinct, group by, agregasi, dan teknik seperti pada SQL. Hanya saja dilakukan pada tabel yang dimuat dari file ke RAM.
\subsection{Bukti Plagiarisme}
\begin{figure}[h]
	\includegraphics[width=10cm]{figures/dzihan/bukti.png}
	\centering
\end{figure}
%%%%%%%%%%%%%%%%%%%%%%%%%%%%%%%%%%%%%%%%%%%%%%%%%%%%%%%%

\section{ Dwi Septiani Tsaniyah }
\begin{enumerate}
\item Apa itu fungsi file csv, jelaskan sejarah dan contoh
File CSV (Nilai Berbatas Koma) adalah tipe file khusus yang dapat Anda buat atau edit di Excel. File CSV menyimpan informasi yang dipisahkan oleh koma, bukan menyimpan informasi dalam kolom. Saat teks dan angka disimpan dalam file CSV, mudah untuk memindahkannya dari satu program ke program lain. Misalnya, Anda dapat mengekspor kontak dari Google ke dalam file CSV, kemudian mengimpornya ke Outlook.
Creating Shared Value (CSV) adalah sebuah konsep dalam strategi bisnis yang menekankan pentingnya memasukkan masalah dan kebutuhan sosial dalam perancangan strategi perusahaan. CSV merupakan pengembangan dari konsep tanggung jawab sosial perusahaan (Corporate social responsibility, CSR). Konsep ini pertama kali diperkenalkan oleh Michael Porter dan Mark Kramer pada tahun 2006. Konsep CSV didasari pada ide adanya hubungan interdependen antara bisnis dan kesejahteraan sosial. Porter mengkritik bahwa selama ini bisnis dan kesejahteraan sosial selalu ditempatkan berseberangan. Pebisnis pun rela mengorbankan kesejahteraan sosial demi keuntungan semata, misalnya dengan melakukan proses produksi yang tidak memperhatikan lingkungan atau menciptakan polusi. CSV menekankan adanya peluang untuk membangun keunggulan kompetitif dengan cara memasukan masalah sosial sebagai bahan pertimbangan utama dalam merancang strategi perusahaan.
contoh : Ketika Toyota memperkenalkan Prius, sebuah kendaraan hybrid listrik/bensin, Toyota berhasil mendapatkan keunggulan kompetitif dengan memasarkan sebuah kendaraan yang tidak hanya memberikan keuntungan ekonomis, namun juga berdampak positif bagi lingkugan. Urbi, sebuah perusahaan konstruksi asal Meksiko, mengembangkan pasar perumahan dengan memberikan kredit murah untuk pekerja dengan gaji kecil, Whole Foods Market telah menjadi pemimpin kategori di segmen supermarket dengan menawarkan makanan organik dan alami kepada konsumen yang sadar lingkungan. Perusahaan juga dapat meningkatkan keunggulan kompetitif dengan melakukan investasi di komunitas di mana mereka beroperasi. Nestlé, misalnya, berhubungan sangat dekat dengan Distrik Susu Moga di India, melakukan investasi pada infrastruktur lokal, dan mentransfer teknologi kelas dunia untuk membangun rantai suplai yang kompetitif sekaligus meningkatkan kesejahteraan sosial melalui peningkatan kesehatan masyarakat, pendidikan yang lebih baik, dan pertumbuhan ekonomi.
\item Aplikasi-aplikasi apa saja yang bisa menciptakan file csv
\begin{itemize}
\item Texteditor , Seperti notepad++,visual studio code,atom,sublime dan lain sebagainya
\item Program Spreadsheet , Seperti excell,google spreadshare,LibreOfficecalc
\end{itemize}
\item Jelaskan bagaimana cara menulis dan membaca file 
 Ada dua cara untuk mengimpor data dari file teks dengan Excel dapat membukanya di Excel, atau mengimpornya sebagai rentang data eksternal. Untuk mengekspor data dari Excel menjadi file teks, gunakan perintah Simpan Sebagai dan ubah tipe file dari menu menurun.
Ada dua format file teks yang biasanya digunakan:
File teks berbatas (.txt), dengan karakter TAB (kode karakter ASCII 009) yang biasanya memisahkan setiap bidang teks. 
File teks nilai yang dipisahkan koma (.csv), dengan karakter koma (,) yang biasanya memisahkan setiap bidang teks.
\item Jelaskan sejarah library csv
library csv dibuat untuk permudah mengolah data. Dan mempermudah untuk melakukan export dan import file csv itu sendiri
\item Jelaskan sejarah library pandas
Pandas merupakan tool yang dapat digunakan sebagai alat analisis data dan struktur untuk bahasa pemrograman Python. Pandas dapat mengolah data dengan mudah, salah satu fitur yang ada dalam pandas adalah Dataframe. 
\item Jelaskan fungsi-fungsi yang terdapat di library csv
Terdapat 2 fungsi yang bisa digunakan oleh library csv
Pertama,fungsi membaca file csv.
fungsi ini bisa menggunakan list dan dictionary
Dengan list :
\lstinputlisting[firstline=11, lastline=21]{src/1174027/1174027_csv.py}
Dengan dictionary :
\lstinputlisting[firstline=24, lastline=33]{src/1174027/1174027_csv.py}
Kedua,fungsi menulis file csv.
\lstinputlisting[firstline=36, lastline=40]{src/1174027/1174027_csv.py}
\item Jelaskan fungsi-fungsi yang terdapat di library pandas
Hampir sama dengan library,akan tetapi library pandas penulisannya lebih sederhana di banding library csv dan library pandas terlihat lebih rapih dibanding library csv.
\lstinputlisting[firstline=43, lastline=44]{src/1174027/1174027_csv.py}
\end{enumerate}

%%%%%%%%%%%%%%%%%%%%%%%%%%%%%%%%%%%%%%%%%%%%%%%%%%%%%%%%%%%
\section{Choirul Anam}
\begin{enumerate}
    \item Apa itu fungsi file csv, jelaskan sejarah dan contoh
    CSV adalah suatu format data dalam basis data dimana setiap record di pisahkan dengan tanda koma (,) atau titik koma (;). File CSV dapat dibuka dengan berbagai text editor contohnya seperti Notepad, Wordpad bahkan Microsoft Excel.
    Dari rilis pertama, Excel menggunakan format file biner yang disebut Binary Interchange File Format (BIFF) sebagai format file utamanya. Ini berubah ketika Microsoft merilis Office System 2007 yang memperkenalkan Office Open XML sebagai format file utamanya. Office Open XML adalah file kontainer berbasis XML yang mirip dengan XML Spreadsheets (XMLSS), yang diperkenalkan di Excel 2002. File versi XML tidak bisa menyimpan makro VBA.
    Meskipun mendukung format XML baru, Excel 2007 masih mendukung format lama yang masih berbasis BIFF tradisional. Selain itu Microsoft Excel juga mendukung format Comma Separated Values (CSV), DBase File (DBF), SYMbolic LinK (SYLK), Format Interchange Data (DIF) dan banyak format lainnya, termasuk format lembar kerja 1-2 Lotus - 3 (WKS, WK1, WK2, dll.) Dan Quattro Pro.
    \item Aplikasi-aplikasi apa saja yang bisa menciptakan file csv
    \begin{itemize}
        \item Texteditor
        Seperti notepad++,visual studio code,atom,sublime dan lain sebagainya
        \item Program Spreadsheet
        Seperti excell,google spreadshare,LibreOfficecalc
    \end{itemize}
    \item Jelaskan bagaimana cara menulis dan membaca file csv di excel atau spreadsheet
    Untuk menulisnya di baris pertama buat headernyalalu di baris kedua sampai kebawahnya itu untuk data, lalu di save.
    dan untuk membukan atau membaca file csv tersebut pergi ke file csv lalu double klik pada file tersebut.
    \item Jelaskan sejarah library csv
    library csv rancang untuk permudah dalam mengolah data. Dan untuk mempermudah melakukan export dan import file csv tersebut.
    \item Jelaskan sejarah library pandas
    library pandas dibuat agar bahasa pemograman python bisa bersaing R dan matlab, yang digunakan untuk mengolah banyak data , keperluan big data, data mining data science dan sebagainya.
    \item Jelaskan fungsi-fungsi yang terdapat di library csv
    Terdapat 2 fungsi yang bisa digunakan oleh library csv
    Pertama,fungsi membaca file csv atau reader
    yang kedua menulis file csv atau dict.reader
   \item Jelaskan fungsi-fungsi yang terdapat pada library pandas
   pertama yaitu ada fungsi head dan tail diamana fungsi ini digunakan untuk melihat sample data
   yang kedua ada fungsi add dimana digunakan untuk menambah data. 
\end{enumerate}
%%%%%%%%%%%%%%%%%%%%%%%%%%%%%%%%%%%%%%%%%%%%%%%%%%%%%%%%
